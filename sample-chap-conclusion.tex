%!TEX ROOT = thesis.tex
\chapter{Conclusion}
\section{Introduction}
%In this last chapter, you may outline the success of your project when compared to the objectives that were set. You may suggest further work for your research area.
In the current project phase, normal driver driving behavior profiling can be achieved by using the KNIME. However, the method described in this project is needed to be enhanced. A model should be trained for machine learning. In the future, a new driver vehicle operation data will be able to be classified without clustering. 

\section{Conclusion}

Driver driving behavior profiling using vehicle on board diagnostic (OBD) information and K Means Algorithm is described in this project. OBD interface and smartphone are utilized to collect vehicle operation data and GPS data. According to the value of linear correlation among the features of the vehicle operation data, some of the features are selected to be used in K Means Algorithm for clustering. The selected features contain GPS Altitude, GPS Bearing, GPS Vehicle Speed (km/h), Engine Speed, Throttle Position, and Speed Test. 

Each vehicle operation record is labeled as good, medium or bad condition according to the clusters. Each driver will be categorized to three groups based on the number of each condition the driver had. In the next project phase, this method will be implemented in order to train the machine to classify the drivers.
%A good final year report should summarize the most important findings and conclude. Always make explanations complete. Avoid speculation that cannot be tested in the foreseeable future. Discuss possible reasons for expected or unexpected findings. 