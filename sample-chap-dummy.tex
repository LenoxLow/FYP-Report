%!TEX ROOT = thesis.tex
\chapter{Research Methodology}

\section{Driving Operation Data Acquisition}
For this project, ELM327 OBD-II Bluetooth Car Diagnostic Interface is required to collect the vehicle operation data. The device needs to be connected with the OBD-II socket of the vehicle. Few drivers had been invited to drive the vehicle that connected with the device. 

A smartphone that installed Torque Lite version Android application is needed to collect the data. The drivers need to put the particular smartphone inside the vehicle while they are driving. After the driving session, a CSV file will be generated by the Torque and stored in the file directory of the smartphone. The CSV file will be uploaded to the Google Drive. In the end, the Google Drive stored few CSV files after different drivers' driving session.

For this project, the vehicle OBD-II information data and Global Positioning System (GPS) data need to be collected. The vehicle OBD-II information data include the vehicle speed, engine speed, engine coolant temperature, throttle position, and acceleration sensor. The GPS data include latitude coordinate, longitude coordinate, GPS speed, GPS altitude, horizontal dilution of precision, GPS bearing, and GPS acceleration.

\section{Driving Operation Data Preprocessing}
The first 30 rows and last 30 rows of the collected vehicle operation data need to be unselected. It is because drivers just come out from the parking at the begining or drive into a parking slot at the end of the driving session. 

\section{Feature Derivation}
In this project, features have been derived from the collected vehicle operation data and GPS data. The first derived feature is speed limit condition. The speed limit condition is an attribute to define whether the driver exceeded the speed limit at the particular location and time. The value of the speed limit condition is "Yes" or "No".

In order to get the value of speed limit condition, the road condition needs to be specified first. The road condition can be specified by using GPS Bearing and GPS coordination. The change rate of the GPS Bearing can be used to determine the driver is driving straight or turning. Different areas have different speed limitations. The GPS coordination is used for determining the driving area. The speed limit of the roads will be used for determining the speed limit condition.
