%!TEX ROOT = thesis.tex
\chapter{Design}

\section{Driving Operation Data Acquisition}
In this project, few drivers have been invited to participate in the data collection. The ELM327 device needs to be connected withe the vehicle OBD-II socket. The smartphone with the Torque(Lite) needs to be put in the vehicle. It needs to connect with the ELM327 device and GPS location service. Torque(Lite) logging function needs to be triggered and stopped in each driver driving session. 

Vehicle operation data will be recorded in every second. The drivers is requested to drive at least 10 mininutes. At least 600 vehicle operation records can be collected from each driver. All drivers was driving in different path.

For this project, the vehicle OBD-II information data and GPS data need to be collected. The vehicle OBD-II information data include the vehicle speed, engine speed, engine coolant temperature, throttle position, and acceleration sensor. The GPS data include latitude coordinate, longitude coordinate, GPS speed, GPS altitude, horizontal dilution of precision, GPS bearing, and GPS acceleration.

Speed limit condition is a value to determine whether the driver exceeded the speed limit at the particular time frame. The value of the speed limit condition is '0' or '1'. '0' means the driver exceeded the speed limit at the particular time frame. '1' is on the contrast. The speed limit will be determined by the road condition. The road condition includes straight road, curve road, traffic light intersection, intersection, roundabout, and state road. 

In Malaysia, speed limit of expressways by default is 110 km/h, but it will be 90km/h in crosswind area, mountainous stretches area, and urban area. Speed limit of state road or federal road is 80km/h, town area will be 60km/h. Due to some road condition, a 50km/h speed limit sign board will be placed in the area of having curve road. 

Drivers suppose to stop at the intersection and observe the surroundings vehicle movement before turning to the another route. So, 40km/h speed limit will be set in this project to determine the drivers' speed limit condition. The intersection having traffic light will be applied with 40km/h speed limit also. In some circunstances, the driver will increase the vehicle speed when the traffic light of the intersection turned to amber signal. The driver will stop the car uncomfortably due to the insufficient time of amber period. The driver need to make decision at the stop-line either to pass through the intersection after red signals or brake hard in front of the intersection. This action will increase the potential of accident occurances.\cite{kulanthayan:phang:hayati:2007}

\section{Driving Operation Data Preprocessing}
The first 30 rows and last 30 rows of the collected vehicle operation data need to be unselected. It is because drivers just come out from the parking at the begining or drive into a parking slot at the end of the driving session. The unselected data is always incomplete.

\section{Data Fusion}
All the pieces of preprocessed data for each driver are required to be concatenated as a big table before performing clustering.

\section{Establish the driving operation model by K Means Algorithmn}
A workflow will be designed and implemented by using KNIME Analytics Platform. The workflow will be able to input the vehicle oepration data file and perform K Means Algorithm on the dataset. Three clusters will be identified through the workflow. The three clusters will represent the high risk, medium risk, and low risk vehicle operation. The labeled dataset will be separated according to the drivers and output each driver vehicle operation dataset accordingly.  

\subsection{K Mean Algorithm}



\section{Driving Behavior Analysis}
In this project, the driver driving behavior will be determined by the mean of the labels of the dataset. The result will be visualized by using the Tableau.